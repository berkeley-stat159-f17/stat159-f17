% Document settings
\documentclass[11pt]{article}
\usepackage[margin=1in]{geometry}
\usepackage[pdftex]{graphicx}
\usepackage[utf8]{inputenc}
\usepackage{multirow}
\usepackage{setspace}

%-----------------------------------------------------------------------------
% The hyperref package gives us a pdf with properly built
% internal navigation ('pdf bookmarks' for the table of contents,
% internal cross-reference links, web links for URLs, etc.)
\usepackage{hyperref}
\hypersetup{pdftex,  % needed for pdflatex
  breaklinks=true,  % so long urls are correctly broken across lines
  colorlinks=true,
  urlcolor=blue,
  linkcolor=darkorange,
  citecolor=darkgreen,
  }

\pagestyle{plain}
\setlength\parindent{0pt}

\begin{document}

% Course information
\begin{tabular}{ l l }
  \multirow{3}{*}{\includegraphics[height=1.25in,width=1.25in]{ucberkeleyseal_874_540}}
  & \LARGE Statistics 159 \& 259 --- Fall 2017 Syllabus\\
  & \LARGE Reproducible and Collaborative Statistical Data Science \\\\
%  & \Large CCN: 87680 (Stat 159) and 87812 (Stat 259)\\
  & \Large Class meets TuTh 9:30--11A in 3106 Etcheverry \\
  & \Large Lab meets W 9--11A or 11--1P in 330 Evans \\\\
\end{tabular}
\vspace{10mm}

% Professor information
\begin{tabular}{ l l }
  \multirow{6}{*}
  & \large Professor: Fernando Pérez, \url{http://fperez.org} \\
  & \large Email: \tt{fernando.perez@berkeley.edu} \\
  & \large Office Location: 419 Evans Hall \\
  & \large Office Hours: Tu 11A and W3P in 419 Evans \\
  & \large GSI: Elijahu Ben-Michael \\
  & \large GSI Office Hours: Mon 9-11A, Fri 3-5P, both in 342 Evans \\
  & \large Email: \tt{ebenmichael@berkeley.edu} \\
\end{tabular}
\vspace{5mm}
\begin{center} I reserve the right to make changes to the syllabus.\\
\end{center}

% Course details
\textbf {\large \\ Course Description:}
A project-based introduction to statistical data science. Through lectures,
computational laboratories, readings, homeworks, and a group project, you will learn
practical techniques and tools for producing statistically sound and
appropriate, reproducible, and verifiable computational answers to scientific
questions.  The course emphasizes version control, testing, process automation,
code review, and collaborative programming. Software tools include Bash, Git,
Python, Jupyter and \LaTeX.

\textbf {Prerequisites:} Statistics 133, Statistics 134, and Statistics 135
(or equivalent). Graduate standing is required to register for Statistics 259.

\textbf {Credit Hours:} 4 \\

\textbf {\large Text(s):}
Readings will be assigned weekly and will mostly consist of articles and tutorials. \\

\textbf {\large Course Objectives:} \\
At the completion of this course, students will:
\begin{enumerate} \itemsep-0.4em
  \item understand the issues regarding reproducible research in modern scientific practice, including the definitions of key concepts and the different challenges that exist across disciplines
  \item understand the computational and statistical issues involved with
  reproducibility
  \item be proficient at the Unix commandline
  \item be proficient at version control with Git
  \item be able to write documents in Markdown or \LaTeX\ (including using
        pandoc)
  \item be familiar with scientific computing in Python


\end{enumerate}

\textbf {\large Grading:} \\
\hspace*{40mm}
\begin{tabular}{ l l }
Reading  & 10\% \\
Quiz  & 10\% \\
Homework & 10\% \\
Project 1 & 10\% \\
Project 2 & 20\% \\
Project 3 & 40\% \\
\end{tabular} \\\\

\textbf{Readings:} For each assigned reading, you will submit a 2 paragraph
report by 21:00 on the Thursday it is due. The first paragraph should summarize
the reading.  The second paragraph should briefly explore something that
interested you (e.g., you may wish to focus on one aspect of the paper in more
depth, you may wish to discuss something in the reading that you disagree with).\\

\textbf{Quizzes:} Quizzes will be held during class or lab.  I will drop your
lowest score.\\

\textbf{Homework:} There will be small individual homework assignments (to be
submitted by 21:00 on the Thursday it is due) as well as three team projects.
You may discuss the homework assignments with classmates, but you will be
required to work on the homework independently and prepare an individual
submission.\\


\textbf{Projects:} During the course, you will work on three, increasingly
complex projects:

1. In pairs, you will attempt to replicate a paper from the scientific
literature, will document your process, and will write a short report on your
process and findings.  I will provide you with suggestions for what journals and
papers to consider, but you will have freedom to choose one that interests you.

2. In teams of three, you will complete the analysis of a dataset and will produce both a written report as well as a set of artifacts (code, figures, etc.) that should be fully reproducible.  You will choose which dataset to work on from a few options given by the instructor.

3. Final project: in teams of five, you will work on an in-depth replication and post-publication peer-review of an existing paper in the literature.  All teams will  work on the same paper, but independently of each other.


For the first project you can choose your own work partner; for the second and third, I will make the team assignments. All members of any team must be familiar with the entirety of the work, I may call upon any member to discuss any aspect of the work and you should be able to demonstrate reasonable familiarity with areas you didn't work on, as well as complete expertise on the aspects you contriubted to.\\

%\newpage

\textbf{\large Course Policies:}

\textbf{Attendance and behavior in class}: You are expected to attend all lectures
and labs.  Any known or potential extracurricular conflicts should be discussed
in person with me during the first two weeks of the semester, or as
soon as they arise. \textbf{Cellphones} are to be silenced during class time and should not be used at all (if you have an emergency, step outside of the classroom to handle it and notify me afterwards).
\textbf{Laptop} use during class will often be required, but should be
used for course work only (i.e., not for surfing the web).\\

\textbf{Submission of assignments}: Assignments will be accepted by electronic
submission to GitHub only.  There will be no makeup quizzes. No
late reading reports or homeworks will be accepted. \\ % Grades of Incomplete will be granted
%only for dire medical or personal emergencies that cause you to miss the final project
%presentation, and only if your work up to that point has been satisfactory.\\

\textbf{Academic integrity}: Any test, paper, or report submitted by you is
presumed to be your own original work that has not previously been submitted for
credit in another course. While you are encouraged to work together on homework
assignments, the work and writeup must be your own. For example, suggesting a
function to another student is acceptable, whereas simply giving him or her your
own code is not.  If you are not clear about the expectations for completing an
assignment or taking a quiz, be sure to seek clarification from me or the GSI
beforehand. Any evidence of cheating and plagiarism will be subject to
disciplinary action.  Please read the (brief) Honor Code
(\url{http://teaching.berkeley.edu/berkeley-honor-code}) and its accompanying discussion carefully.  Ask me if you have any questions.\\

\textbf{Class discussion}:
Rather than emailing questions to the teaching staff, you should post
your questions on Piazza (the class page is at:
\url{https://piazza.com/berkeley/fall2017/stat159259/home}).  When asking
questions, especially regarding problems with code, make every effort to be
clear and to provide sufficient information for others to be able to understand
the problem you are having.  This may include providing a minimal amount of code
to replicate your problem, or copies, including screenshots, of the results you
are getting.  You may find this document useful, it contains tips on how to
phrase questions regarding code and computation in an effective manner:
\url{https://www.mikeash.com/getting_answers.html}.\\


\textbf{Students with disabilities}: If you need accommodations, please make
arrangements in at timely manner through DSP. If your DSP oficer has already communicated your accommodation letter to me, I will have it in my files.  I am happy to talk privately to you to work out the specifics of your situation, to ensure you have the support you need.\\

\textbf{A note on Hurricane Harvey}: if you or your loved ones have been affected by the situation unfolding in Texas and Louisiana, I am happy to make necessary accommodations. Contact me for an appointment and we can discuss the matter privately.

% \noindent\textbf{Important Dates}:
% \begin{center} \begin{minipage}{5in}
% \begin{flushleft}
% Form teams \dotfill Sept. 22\\
% Homework 1 \dotfill Sept. 24\\
% Project proposal \dotfill Oct. 1\\
% Homework 2 \dotfill Oct. 26\\
% Progress presentation \dotfill Nov. 12\\
% Draft report \dotfill Nov. 12\\
% BIC tour \dotfill Nov. 23\\
% Project presentation \dotfill Dec. 1 \& 3\\
% Final report \dotfill Dec. 14\\
% \end{flushleft}
% \end{minipage}
% \end{center}

% \newpage

% % Course Outline
% \textbf {\large Tentative Course Outline}:

% The weekly coverage might change as it depends on the progress of the class.
% %However, you must keep up with the reading assignments.

% \begin{table}[h!]
% \normalsize % The size of the table text can be changed depending on content. Remove if desired.
% \begin{tabular}{ | c | c | }
% \hline
% \textbf{Week} & \textbf{Content} \\
% \hline
% Week 1 & \begin{minipage}{.85\textwidth}
% \begin{itemize} \itemsep-0.4em
% 	\vspace{1mm}
% 	\item Course introduction
% 	\vspace{1mm}
% \end{itemize}
% \end{minipage} \\
% \hline
% Week 2 & \begin{minipage}{.85\textwidth}
% \begin{itemize} \itemsep-0.4em
% 	\vspace{1mm}
% 	\item Introduction to Git; Using the bash shell
% 	\item \textbf{Reading 1}: \href{https://osf.io/zqbu2}{L Preeyanon, AB Pyrkosz, and CT Brown.
%              ``Reproducible bioinformatics research for biologists.''
%              Implementing Reproducible Research (2014)}
% 	\vspace{1mm}
% \end{itemize}
% \end{minipage} \\
% \hline
% Week 3 & \begin{minipage}{.85\textwidth}
% \begin{itemize} \itemsep-0.4em
% 	\vspace{1mm}
% 	\item Statistical analysis of fMRI
% 	\item \textbf{Reading 2}: \href{http://arxiv.org/pdf/0906.3662v1}{MA Lindquist. ``The statistical analysis of fMRI data.''}
%               %Statistical Science 23.4 (2008)
%               (2008)
% 	\vspace{1mm}
% \end{itemize}
% \end{minipage} \\
% \hline
% Week 4 & \begin{minipage}{.85\textwidth}
% \begin{itemize} \itemsep-0.4em
% 	\vspace{1mm}
% 	\item Introduction to Python
% 	\item \textbf{Reading 3}: \href{http://www.computer.org/cms/Computer.org/ComputingNow/issues/2015/04/T-mcs2011020013.pdf}{F P\'{e}rez, BE Granger, and JD Hunter.
%               ``Python: an ecosystem for scientific computing.''}
%               %Computing in Science \& Engineering 13.2 (2011)
%               (2011)
% 	\vspace{1mm}
% \end{itemize}
% \end{minipage} \\
% \hline
% Week 5 & \begin{minipage}{.85\textwidth}
% \begin{itemize} \itemsep-0.4em
% 	\vspace{1mm}
% 	\item Scientific computing with Python
%         \item \textbf{Form teams}
% 	\item \textbf{Homework 1}
% 	\vspace{1mm}
% \end{itemize}
% \end{minipage} \\
% \hline
% Week 6 & \begin{minipage}{.85\textwidth}
% \begin{itemize} \itemsep-0.4em
% 	\vspace{1mm}
% 	\item Collaborative workflow with Git
% 	\item \textbf{Project proposal}
% 	\vspace{1mm}
% \end{itemize}
% \end{minipage} \\
% \hline
% Week 7 & \begin{minipage}{.85\textwidth}
% \begin{itemize} \itemsep-0.4em
% 	\vspace{1mm}
% 	\item Exploratory data analysis
% 	\item \textbf{Reading 4}: \href{http://statweb.stanford.edu/~wavelab/Wavelab_850/wavelab.pdf}{JB Buckheit and DL Donoho.
%               ``Wavelab and reproducible research.'' (1995)}
% 	\vspace{1mm}
% \end{itemize}
% \end{minipage} \\
% \hline
% Week 8 & \begin{minipage}{.85\textwidth}
% \begin{itemize} \itemsep-0.4em
% 	\vspace{1mm}
% 	\item Project organization, process automation
% 	\item \textbf{Reading 5}: \href{http://www.jarrodmillman.com/publications/millman2014developing.pdf}{KJ Millman and F P\'{e}rez.
%               ``Developing open source scientific practice.''}
%               %Implementing Reproducible Research (2014)
%               (2014)
% 	\vspace{1mm}
% \end{itemize}
% \end{minipage} \\
% \hline
% Week 9 & \begin{minipage}{.85\textwidth}
% \begin{itemize} \itemsep-0.4em
% 	\vspace{1mm}
% 	\item Statistical analysis
% 	\item \textbf{Homework 2}
% 	\vspace{1mm}
% \end{itemize}
% \end{minipage} \\
% \hline
% Week 10 & \begin{minipage}{.85\textwidth}
% \begin{itemize} \itemsep-0.4em
% 	\vspace{1mm}
% 	\item TBD % Statistical analysis II
% 	\vspace{1mm}
% \end{itemize}
% \end{minipage} \\
% \hline
% Week 11 & \begin{minipage}{.85\textwidth}
% \begin{itemize} \itemsep-0.4em
%         \vspace{1mm}
%         \item \textbf{Project progress presentation}
%         \vspace{1mm}
% \end{itemize}
% \end{minipage} \\
% \hline
% Week 12 & \begin{minipage}{.85\textwidth}
% \begin{itemize} \itemsep-0.4em
% 	\vspace{1mm}
% 	\item TBD % Scientific computing with Python II
%         \item \textbf{Draft report}
% 	\vspace{1mm}
% \end{itemize}
% \end{minipage} \\
% \hline
% Week 13 & \begin{minipage}{.85\textwidth}
% \begin{itemize} \itemsep-0.4em
% 	\vspace{1mm}
%         \item TBD % Model selection/validation, Selective inference
% 	\vspace{1mm}
% \end{itemize}
% \end{minipage} \\
% \hline
% Week 14 & \begin{minipage}{.85\textwidth}
% \begin{itemize} \itemsep-0.4em
% 	\vspace{1mm}
% 	\item TBD % Final thoughts
% 	\vspace{1mm}
% \end{itemize}
% \end{minipage} \\
% \hline
% Week 15 & \begin{minipage}{.85\textwidth}
% \begin{itemize} \itemsep-0.4em
%         \vspace{1mm}
%         \item \bf{Project presentation}
%         \vspace{1mm}
% \end{itemize}
% \end{minipage} \\
% \hline
% Week 16 & \begin{minipage}{.85\textwidth}
% \begin{itemize} \itemsep-0.4em
%         \vspace{1mm}
%         \item \bf{RR Week}
%         \vspace{1mm}
% \end{itemize}
% \end{minipage} \\
% \hline
% Week 17 & \begin{minipage}{.85\textwidth}
% \begin{itemize} \itemsep-0.4em
%         \vspace{1mm}
%         \item \bf{Final report due Monday}
%         \vspace{1mm}
% \end{itemize}
% \end{minipage} \\
% \hline
% \end{tabular}
% \end{table}

\end{document}
